\documentclass[10pt,a4paper]{article}
\include{structure}
\hyphenation{Some-long-word}
\begin{document} 

\maintitle{Research Summary}
\noindent \\ Ph.D. Student: Alain Galvan \bull Florida International University \bull Advisor: Dr. Francisco Raul Ortega, Ph.D.

\spacedhrule{1em}{0.5em}

%---------------------------------------------------------------------------------
%	SUMMARY SECTION
%---------------------------------------------------------------------------------

\includegraphics[width=0.9\linewidth]{cover.png}
\begin{center}
\roottitle{3D Virtual Programming Language to Increase Number of \\ Minorities in Computer Science Education}
\end{center}
\vspace{1em}

\begin{multicols}{2}

\section{Introduction}

I propose a 3D Virtual Programming Language designed for beginners and intermediate students. The direction of this research has the potential to increase the number of women and minorities in Computer Science. 

Our research lab prepared a user study to gauge the interest and ideas of what figures would work best for a 3D programming system. Our results showed that it is hard for CS students to provide clear 3D representations for programming concepts in some instances, indicating a need for a complete system that is thoroughly tested with user studies.

With the advent of facebook technologies like ReasonML, React, and GraphQL this poses an opportunity to use these technologies in combination with the Khronos Group's Vulkan and GLTF Specification\cite{gltf} to create a high performance, cross platform, real time rendering system for Web/Desktops/Mobile. Such a system could then be used to build a Virtual Reality (VR) programming environment that can be expanded upon with online collaboration and real time rendering techniques like those of Inigo Quilez' (Graphics Engineer, Oculus) ShaderToy\cite{quilez}.

\section{Overview}

Building Block Programming (BBP), such as MIT's Scratch have proven to be of value for Computer Science (CS) education in K-12 and College. The value of BBP lies not in reduced complexity but in that it provides a learning environment where concepts can be integrated by students at a faster, more satisfactory rate.

It has been shown that BBP has improved recruitment and retention of women and minorities in CS. Research has shown the importance of diversity in that it promotes greater creativity, better decision-making, and outcomes. Nonetheless, the number of women (as well as minorities) in CS remain very low in spite of the fact that women represent more than 50\% of the students enrolled in post-secondary institutions since 1980 \cite{nces}. 

Furthermore, CS had a larger representation of women until 1983/1984, when numbers began to decline and they have continued to fall ever since – with a current average of 14.1\% in the United States. In search of a way to include more people into CS, we have started developing a virtual reality solution for K-12 and College level education (with emphasis on the latter population) using BBP. We called our approach a 3D Virtual Programming Language (3DVPL) as a means to provide an interactive tool for beginners and intermediate students. Our 3D-VPL language objective is to generate Python code that can construct small 3D animated applications in the same environment. One of the biggest challenges we face is interpreting different concepts of a programming language in a 3D environment. This research study provides an early attempt to understand the ideal objects to be used based on how CS students visualize programming concepts, in order to improve our 3D-VPL. Research done in different schools has shown that re-designing introductory courses improves womens recruitment and retention in CS, such as a study performed at Georgia Tech \cite{forte}.

Such a programming environment has the potential to increase the number of women and minorities in compute, and would be in Facebook's best interest to support.



\bibliographystyle{abbrv}

\bibliography{paper}

\end{multicols}

\section{Appendix | Figures}
\begin{center}
\includegraphics[width=0.9\linewidth]{running-program.png}

\vspace{1em}

\includegraphics[width=0.9\linewidth]{inside-class.png}

\vspace{1em}


\includegraphics[width=0.9\linewidth]{vrprogramming.eps}
\end{center}

\end{document}
\documentclass[10pt,a4paper]{article}
\include{structure}
\hyphenation{Some-long-word}
\begin{document} 

\maintitle{\includegraphics[scale=0.03]{../../appicon.jpg} Alain Galvan}
\noindent\href{mailto:hi@Alain.xyz}{Hi@Alain.xyz}\bull
\textsmaller{+}1 (305) 302-9275 \bull
\href{https://alain.xyz}{Alain.xyz} \bull Miami, United States

\spacedhrule{0.9em}{-0.4em}

%---------------------------------------------------------------------------------
%	SUMMARY SECTION
%---------------------------------------------------------------------------------

\begin{multicols}{2}

\roottitle{Research Summary}

\noindent \textit{I'm a graduate research assistant for \acr{FIU}'s OpenHID Lab, a Human Computer Interaction (\acr{HCI}) lab part of the High Performance Database Research Center (\acr{HPDRC}), where my research focuses on low level graphics programming, in addition to solving graphics related problems for other researchers in the lab. \\ \\ I'm currently pursuing a PhD in Computer Science with a focus in Computer Graphics Software Architecture.} \\



\end{multicols}


\roottitle{High Performance and Scalable Real Time Clientside Rendering with Centralized Store and Transmissible Format}

With the advent of high performance cross platform facebook technologies like ReasonML and GraphQL this poses an opourtunity to use these technologies in combination with the Khronos Group's Vulkan and GLTF Specification to create a high performance real time rendering system for web browsers.

Flipboard's research into high performance view rendering noted that relying on standard HTML view formats proved too slow for complex animations, thus they built an entire renderer for React using Canvas Rendering.

I propose a similar approach but taking advantage of the WebGPU working group's draft specification, WebGL 2, and Vulkan on desktops.

A node based abstraction whereas users are able to build entire complex scenes for use in 3D environments such as Facebook's Oculus Rift. 

Inigo Quilez, Oculus Graphics Engineer, John Carmack

\section{Related Work}
\label{sec:relatedwork}

\bibliographystyle{abbrv}
\nocite{*}
\bibliography{paper}

\end{document}